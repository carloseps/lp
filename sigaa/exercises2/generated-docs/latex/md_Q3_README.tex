Para testar o código é necessário apenas que o usuário interaja com os menus que irão aparecer durante o tempo de execução. Tendo a possibilidade de\+:


\begin{DoxyEnumerate}
\item Listar alunos
\item Ordenar alunos
\item Sair do programa
\end{DoxyEnumerate}

É importante ressaltar que ao selecionar a opção \char`\"{}$\ast$2 -\/ Ordenar alunos$\ast$\char`\"{} abre-\/se um novo menu que disponibiliza opções de ordenação da lista de alunos. No caso, por qual campo eles devem ser ordenados, sendo\+:


\begin{DoxyEnumerate}
\item Matrícula
\item Nome
\item Nota
\end{DoxyEnumerate}

Existe também a opção \char`\"{}$\ast$4 -\/ Sair$\ast$\char`\"{}, que se selecionada, faz com que retorne ao menu inicial.

PS\+: Não é necessário preencher dados e informações sobre os alunos, como\+: matrícula, nome e notas. O código conta com um método que popula a lista de alunos sozinho.

São criados alunos com nomes pertencentes a um array de nomes, que possui 100 strings, ou seja, 100 opções de nomes aleatórios de pessoas comuns.

A matrícula é gerada através de uma estratégia de multiplicação de iterador com soma.

As notas de cada aluno são geradas aleatoriamente usando a função {\itshape rand}, que é uma função da biblioteca padrão {\itshape cstdlib} que gera um número pseudo-\/aleatório. 